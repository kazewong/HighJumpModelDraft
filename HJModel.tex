\documentclass[floats,floatfix,showpacs,amssymb,prd,twocolumn,superscriptaddress,nofootinbib]{revtex4-1}

\usepackage{amssymb,amsmath,verbatim,mathtools,needspace,enumitem,etoolbox,graphicx,physics,microtype,afterpage,bm}

\usepackage[dvipsnames, usenames]{xcolor}

\definecolor{linkcolor}{rgb}{0.0,0.3,0.5}
\usepackage[unicode, colorlinks=true, linkcolor=linkcolor, citecolor=linkcolor, filecolor=linkcolor,urlcolor=linkcolor, pdfusetitle]{hyperref}
\usepackage[all]{hypcap}
\usepackage{appendix}
\usepackage[T1]{fontenc}
\usepackage[utf8]{inputenc}
\usepackage{tabularx}
\usepackage{cleveref}
\usepackage[normalem]{ulem}

%\usepackage{ulem}
\usepackage{float}
\usepackage{orcidlink}

\graphicspath{{./figure/}}

\newcommand{\jhu}{\affiliation{Department of Physics and Astronomy, Johns Hopkins University, 3400 N. Charles
Street, Baltimore, MD 21218, USA}}

\definecolor{rb4}{HTML}{27408B}

\newcommand{\kw}[1]{{\color{rb4}[KW: #1 ]}}

\begin{document}

\title{A computational model for the hinge moment in high jump}

\author{Eli}
\jhu
\author{Elia}
\jhu
\author{Kaze W. K. Wong \orcidlink{0000-0001-8432-7788}}
\jhu
%\email{kazewong@jhu.edu}


\date{\today}

\begin{abstract}
tes
\end{abstract}

\maketitle

%\tableofcontents


%%%%%%%%%%%%%%%%%%%%%%
\section{Introduction}

The rest of this paper is organized as follow:
%%%%%%%%%%%%%%%%%%%%%%
\end{document}