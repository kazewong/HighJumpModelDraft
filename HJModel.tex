\documentclass[floats,floatfix,showpacs,amssymb,prd,twocolumn,superscriptaddress,nofootinbib]{revtex4-1}

\usepackage{amssymb,amsmath,verbatim,mathtools,needspace,enumitem,etoolbox,graphicx,physics,microtype,afterpage,bm}

\usepackage[dvipsnames, usenames]{xcolor}

\definecolor{linkcolor}{rgb}{0.0,0.3,0.5}
\usepackage[unicode, colorlinks=true, linkcolor=linkcolor, citecolor=linkcolor, filecolor=linkcolor,urlcolor=linkcolor, pdfusetitle]{hyperref}
\usepackage[all]{hypcap}
\usepackage{appendix}
\usepackage[T1]{fontenc}
\usepackage[utf8]{inputenc}
\usepackage{tabularx}
\usepackage{cleveref}
\usepackage[normalem]{ulem}

%\usepackage{ulem}
\usepackage{float}
\usepackage{orcidlink}

\graphicspath{{./figure/}}

\newcommand{\jhu}{\affiliation{Department of Physics and Astronomy, Johns Hopkins University, 3400 N. Charles
Street, Baltimore, MD 21218, USA}}

\definecolor{rb4}{HTML}{27408B}

\newcommand{\kw}[1]{{\color{rb4}[KW: #1 ]}}

\begin{document}

\title{A computational model for the hinge moment in high jump}

\author{Eli}
\jhu
\author{Elia}
\jhu
\author{Kaze W. K. Wong \orcidlink{0000-0001-8432-7788}}
\jhu
%\email{kazewong@jhu.edu}


\date{\today}

\begin{abstract}
tes
\end{abstract}

\maketitle

%\tableofcontents


%%%%%%%%%%%%%%%%%%%%%%
\section{Introduction}
\label{sec:intro}

% Most study in high jump has been emprical
While there

The rest of this paper is organized as follow:
In Section \ref{sec:method}, we describe the tools we use to construct the high jump models and the assumptions going into the models.
In Section \ref{sec:results}, we show the experiment results.
In Section \ref{sec:discussion}, we discuss the implications of our results and future directions.

\section{Method}
\label{sec:method}

\subsection{Mujoco}

% Describe Mujoco and model

% We choose two simpler model for interpretable results.
While it is possible to integrate a full musculoskeletal model in Mujoco, the complexity of such a model makes it difficult to interpret the results.

We construct two simple models to understand the zeroth order and first order effects in high jump instead. The first model is a rigid stick model, which is a simple model of a rigid rod entering at an angle with a certain velocity. This model gives us insight on how the basic kinematics variable such as incident angle, lean angle, and initial velocity affect the resulting height. The second model is a multi-segment model aims to represent an oversimplified version of the human body. Building on top of the understanding from the first model, we aim to use this model to investigate the criteria for optimal kinematic chain activation.

\subsection{Rigid stick model}

% Define the model and its parameters

\subsection{Multi-segment model}

% Define the model and its parameters

\section{Results}
\label{sec:results}

\subsection{Rigid stick model}

\subsection{Multi-segment model force}

\subsection{Kinematic chain activation}

\section{Discussion}
\label{sec:discussion}


%%%%%%%%%%%%%%%%%%%%%%
\end{document}