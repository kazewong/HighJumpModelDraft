% Define document class
\documentclass[twocolumn]{aastex631}
\usepackage[ruled,vlined,linesnumbered]{algorithm2e}
\usepackage{algorithmic}
\usepackage{color}
\usepackage{booktabs}
\usepackage{xspace}
\usepackage{orcidlink}

\graphicspath{{./figure/}}

\newcommand{\jhu}{\affiliation{Department of Physics and Astronomy, Johns Hopkins University, 3400 N. Charles
Street, Baltimore, MD 21218, USA}}

\definecolor{rb4}{HTML}{27408B}

\newcommand{\kw}[1]{{\color{rb4}[KW: #1 ]}}

\begin{document}

\title{A computational model for the hinge moment in high jump}

\author{Eli Gitter-Dentz$^*$}
\jhu
\author{Elia Janes$^*$}
\jhu
\author{Kaze W. K. Wong \orcidlink{0000-0001-8432-7788}}
\jhu
%\email{kazewong@jhu.edu}
%

\footnote{$^*$ Equal contribution}
\date{\today}

\begin{abstract}
We employ two simple computational models to understand the biomechanics of high jump.
We find xxx
\end{abstract}


%\tableofcontents

%%%%%%%%%%%%%%%%%%%%%%
\section{Introduction}
\label{sec:intro}

% Most study in high jump has been emprical
There have been many empirical studies based on clinical trials \cite{WILSON20112207, MateosPadorno2021KinematicAO, }.

\cite{Sado2021CurvedAI,Fujimori2024JointKD}

The rest of this paper is organized as follow:
In Section \ref{sec:method}, we describe the tools we use to construct the high jump models and the assumptions going into the models.
In Section \ref{sec:results}, we show the experiment results.
In Section \ref{sec:discussion}, we discuss the implications of our results and future directions.

\section{Simulation models}
\label{sec:method}


To simulate the target systems, we employ \textsc{MuJoCo} \cite{todorov2012mujoco} as the underlying engine. 
\textsc{MuJoCo} (Multi-Joint dynamics with Contact) is a physics engine designed for the simulation and control of complex dynamical systems with contacts, constraints, and articulated bodies.
On top of being a physics engine, \textsc{MuJoCo} supports modern features such as differentiable simulation and GPU acceleration.
Because of these features, it is widely used in robotics research and has been the backbone of many state-of-the-art machine learning studies and applications in robotics, such as reinforcement learning and direct control.\cite{1606.01540,mujoco_playground_2025, towers2024gymnasium, Schulman2017ProximalPO}.



% We choose two simpler model for interpretable results.
While it is possible to integrate a full musculoskeletal model in Mujoco, the complexity of such a model makes it difficult to interpret the results.

We construct two simple models to understand the zeroth order and first order effects in high jump instead. The first model is a rigid stick model, which is a simple model of a rigid rod entering at an angle with a certain velocity. This model gives us insight on how the basic kinematics variable such as incident angle, lean angle, and initial velocity affect the resulting height. The second model is a multi-segment model aims to represent an oversimplified version of the human body. Building on top of the understanding from the first model, we aim to use this model to investigate the criteria for optimal kinematic chain activation.

\subsection{Rigid stick model}

% Define the model and its parameters

\subsection{Multi-segment model}

% Define the model and its parameters

\section{Experimental Results}
\label{sec:results}

\subsection{Rigid stick model}

\subsection{Multi-segment model force}

\subsection{Kinematic chain activation}

\section{Discussion}
\label{sec:discussion}

% Investigate the effect of flop

% Employ full body biomechanical model

\bibliography{HJModel}

%%%%%%%%%%%%%%%%%%%%%%
\end{document}
